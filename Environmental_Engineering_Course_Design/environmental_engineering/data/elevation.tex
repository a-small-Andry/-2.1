\subsection{高程图布置}
% \subsubsection{高程布置计算}

污水处理厂厂区设计地坪绝对标高为 498.00 m,污水处理厂进水泵房处污水进水管管底标高为 493.15 m,水泵扬程 $H=8.18$ m。采用 DN600 的管道将污水接入污水厂中格栅前的集水井,充满度为 0.65,水力坡度为 $1\%$。

本次工程只考虑沿程和建筑物的水头损失,其中下式为沿程水力损失计算公式,采用的为 DN500,$i$ 为水力坡降,$1000i$ 查给排水手册表为 $1.39$,$L$ 为管道长度,管道长度即建筑物距离。

\begin{equation}
	H_{\text{沿程}} = i \cdot L
\end{equation}

其中构筑物中粗格栅到提升泵房的损失,粗格栅以及提升泵房是连接在一起的,在前面以及算过,下表直接计算的是细格栅,曝气沉砂池,膜格栅,膜生物反应器,到消毒渠之间的构筑物之间的沿程损失,如表 \ref{tab:Structures along the way} 所示:

\begin{table}[H]
	\centering
	\caption{构筑物沿程}
	\begin{tabular}{ccc}
	\toprule
	构筑物名称 & 构筑物之间的距离(m) & 沿程损失 \\
	\midrule
	细格栅   & 5     & 0.01 \\
	曝气沉砂池 & 10     & 0.14 \\
	消毒池   & 10    & 0.14 \\
	\bottomrule
	\end{tabular}
	\label{tab:Structures along the way}
\end{table}

构筑物水头损失计算如表 \ref{tab:Elevation arrangement table} 所示:

\begin{table}[H]
	\centering
	\caption{高程布置表}
	\resizebox{\textwidth}{!}{
	\begin{tabular}{cccc}
	\toprule
	建筑物名称 & 构筑物及沿程水头损失(m) & 水面上游标高(m) & 水面下游标高(m) \\
	\midrule
	粗格栅   & 0.30  & 493.76 & 493.46  \\
	粗格栅到提升泵房 & 0.05  & 493.46 & 493.41  \\
	提升泵房  & 0.20  & 493.41 & 493.21  \\
	提升泵房到细格栅 & 0.05  & 493.21 & 493.16  \\
	细格栅   & 0.30  & 500.66 & 500.36  \\
	细格栅到沉砂池 & 0.18  & 500.36 & 500.18  \\
	沉砂池   & 0.21  & 500.18 & 499.97  \\
	沉砂池到氧化沟 & 0.04  & 499.97 & 499.93  \\
	氧化沟   & 0.60  & 499.93 & 499.33  \\
	氧化沟到消毒池 & 0.16  & 499.33 & 499.17  \\
	消毒池出水 & 0.00  & 499.17 & 499.17 \\
	\bottomrule
	\end{tabular}}
	\label{tab:Elevation arrangement table}
\end{table}

因为排水河流的最高水位为 494.40 m,消毒池出水为 499.17 m,则它们之间的相对高度为 \eval{$499.17-494.40$} m $>0.5$ m,满足设计排水要求。

