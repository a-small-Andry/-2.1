%----------------------------------------------结论----------------------------------------------%
\null   %--------只是为了空行

\begin{conclusion}
大气污染控制是美丽中国建设的重要组成部分,对锅炉进行烟气处理有着必要性以及紧迫性。在这次课程设计中我对锅炉的烟气排放进行了分析,要确保排放的烟尘符合国家标准,在查找相关文件和文献资料后,我们做出这份课程设计。

首先,我们计算了锅炉的需氧量以及烟气排放量,通过与国家标准对比,发现除尘器的除尘效率需达到$99.391\%$,除硫效率需达到$83.556\%$。为达到除尘效率,我们查阅大量资料,比对多种除尘方式,最终确定以袋式除尘器作为本次设计的除尘器。根据除尘器的除尘效率影响因素,选择圆筒形滤袋及特氟纶(聚四氟乙烯)为滤料、下进气方式、脉冲喷吹的清灰方式的过滤方式。
    
然后,我们再根据所掌握的参数进行了除尘器管道布置的计算、阻力的计算、压力损失的计算以及烟囱设计的计算等。算出烟囱的高度$H=40$m,烟囱出口内径$d=1.2$m,烟囱底部直径$d_1=2.8$m。并根据计算查询设计手册,选定锅炉引风机型号为 \hyperref[tab:The main performance parameters of the selected wind turbine]{G4-73-8D};电动机型号为 \hyperref[tab:Y160M1-2 Motor model parameter table]{Y160M1-2}。

最后,综合以上所有相关设计参数,使用 AutoCAD 制作废气净化系统平面图,并使用\XeLaTeX 编写设计说明书。

\end{conclusion}
%----------------------------------------------结论----------------------------------------------%